\documentclass[10.5pt,letterpaper]{article}
\usepackage[letterpaper,margin=0.5in]{geometry}
\geometry{
 top=10mm,
}
\usepackage[utf8]{inputenc}
\usepackage{mdwlist}
\usepackage[T1]{fontenc}
\usepackage{textcomp}
\usepackage{tgpagella}
\pagestyle{empty}
\setlength{\tabcolsep}{0em}
\usepackage{enumitem}
\usepackage[hidelinks]{hyperref}
\usepackage{fontawesome5}
\usepackage{array}
\setlist[itemize]{leftmargin=1.5em}
% \setlist[itemize]{leftmargin=*}

% indentsection style, used for sections that aren't already in lists
% that need indentation to the level of all text in the document
\newenvironment{indentsection}[1]%
{\begin{list}{}%
    {\setlength{\leftmargin}{#1}}%
    \item[]%
}
{\end{list}}

% opposite of above; bump a section back toward the left margin
\newenvironment{unindentsection}[1]%
{\begin{list}{}%
    {\setlength{\leftmargin}{-0.5#1}}%
    \item[]%
}
{\end{list}}

% format two pieces of text, one left aligned and one right aligned
\newcommand{\headerrow}[2]
{\begin{tabular*}{\linewidth}{l@{\extracolsep{\fill}}r}
    #1 &
    #2 \\
\end{tabular*}}

\begin{document}

% Contact Information
\begin{center}
{\LARGE \textbf{Chengjun Lu}}
\\\vspace{0.5em}
\faPhone\ (217) 418-9736 
\quad \textbullet \quad
\faEnvelope\ \href{mailto:chl253@ucsd.edu}{chl253@ucsd.edu}
% \quad \textbullet \quad
% \faGithub\ \href{https://github.com/NUMPYFFT}{NUMPYFFT}
\end{center}

% Education Section
\subsection*{Education}
\vspace{-0.3em}\hrule\vspace{0.2em}
\begin{itemize}
    \parskip=0.1em

    \item 
    \headerrow
        {\textbf{University of California San Diego}}
        {\textbf{}}
    \\
    \headerrow
        {M.S Electrical and Computer Engineering}
        {\emph{Sep 2023 -- Present}}
        
    \item 
    \headerrow
        {\textbf{University of Illinois at Urbana-Champaign}}
        {\textbf{}}
    \\
    \headerrow
        {B.S Electrical and Computer Engineering (Honor/GPA 3.7)}
        {\emph{Aug 2019 -- May 2023}}

    

    \item
    \headerrow
        {\textbf{Relevant Coursework}}
        {\textbf{}}
    \\
    Algorithm Design and Analysis
    $\cdot$ Numerical Methods
    $\cdot$ Digital Signal Processing
    $\cdot$ Classical AI
    $\cdot$ Machine Learning and Computer Vision
    $\cdot$ Natural Language Processing
    $\cdot$ Robotics and Reinforcement Learning

\end{itemize}

\subsection*{Publications}
\vspace{-0.3em}\hrule\vspace{0.2em}
\begin{itemize}
    \item Rishub Tamirisa, John Won, \textbf{Chengjun Lu}, Ron Arel, Andy Zhou. "FedSelect: Customized Selection of Parameters for Fine-Tuning during Personalized Federated Learning. \textit{ICML Federated Learning Workshop}, 2023
\end{itemize}

% Relevant Experience
\vspace{-1.3em}
\subsection*{Experience}
\vspace{-0.6em}\hrule\vspace{0.2em}

\begin{itemize}
    \item
    \headerrow
        {\textbf{Data-Efficient Robot Learning with SAC Variants}}
        {\emph{April 2024 - August 2024}}
    \begin{itemize*}\vspace{-0.7em}
        \item Accelerated reinforcement learning convergence by augmenting \textbf{Soft Actor-Critic (SAC)} with expert demonstrations collected via \textbf{ManiSkill}, training with a balanced 50:50 mixture of offline and online data.
        \item Engineered a modified \textbf{Twin Delayed DDPG (TD3)} algorithm incorporating \textbf{imitation loss} to bias policy learning, significantly reducing random exploration and outperforming traditional SAC baselines.
    \end{itemize*}

    \item
    \headerrow
        {\textbf{6D Pose Estimation System}}
        {\emph{March 2025 - June 2025}}
    \begin{itemize*}\vspace{-0.7em}
        \item Designed an end-to-end system for robotic grasping that predicts object orientation and position from raw point clouds.
        \item Trained a \textbf{PointNet} architecture for coarse global pose regression and implemented a robust \textbf{3-stage ICP} algorithm for fine local alignment.
        \item Developed a custom \textbf{symmetry-aware loss function} to handle geometric ambiguities for symmetric objects (e.g., cylinders, boxes), improving convergence stability.
        \item Resolved "sliding" issues for planar objects by implementing a hybrid Point-to-Plane and Point-to-Point registration strategy, achieving sub-millimeter residual error ($\sim$1.6mm RMSE).
    \end{itemize*}

    \item
    \headerrow
        {\textbf{A Re-implementation of NeRF from scratch}}
        {\emph{Oct 2023 - Dec 2023}}
    \begin{itemize*}\vspace{-0.7em}
        \item Implemented \textbf{Neural Radiance Fields (NeRF)} from scratch in \textbf{PyTorch}, achieving >30 PSNR on custom datasets without relying on existing NeRF libraries.
        \item Optimized volumetric rendering by implementing patch-based pixel segmentation, achieving a \textbf{2x speedup} in inference time compared to conventional methods.
    \end{itemize*}

    \item
    \headerrow
        {\textbf{Wavelet-Based Feature Space Augmentation for Domain Generalization}} 
        {\emph{Sep 2024 - Dec 2024}}
        
    \begin{itemize*}\vspace{-0.7em}
        \item Introduced a Wavelet-Attention Unit replacing pooling and stride layers with \textbf{Discrete Wavelet Transform (DWT)} for frequency-preserving downsampling, ensuring adaptive feature extraction.
        \item Developed a frequency-space augmentation technique applied to feature representations, improving \textbf{domain generalization} by enhancing robustness against domain shifts with minimal computational overhead.
        \item Achieved a 4.4\% accuracy boost over a baseline \textbf{CNN} and 0.7\% improvement over standard feature-space augmentation on VLCS, validating the effectiveness of frequency-aware learning.
    \end{itemize*}
    
    \item
    \headerrow
        {\textbf{Federated Learning with Lottery Ticket Hypothesis}} 
        {\emph{February 2023 - July 2023}}
    \begin{itemize*}\vspace{-0.7em}
        \item Refined and fine-tuned sub-networks of local clients, termed as non-lottery tickets, integrating these enhancements during multiple communication rounds to strengthen the global model.
        \item Innovatively adjusted client model architectures and parameters throughout the training process, employing a tailored approach to the \textbf{Lottery Ticket Hypothesis} to optimize learning efficacy.
    \end{itemize*}
\end{itemize}


% Skills
\subsection*{Skills}
\vspace{-0.4em}\hrule\vspace{0.3em}
\begin{tabular}{ @{} >{\bfseries}l @{\hspace{6ex}} l }
Programming Languages & Python, MATLAB, SystemVerilog \& Verilog, C/C++, Java \\
Frameworks/Packages & PyTorch, TensorFlow, ROS \\
Spoken Languages & Chinese, English \\
\end{tabular}

% Awards and Honors
% \subsection*{Awards and Honors}
% \vspace{-0.4em}\hrule\vspace{0.3em}
% Bradley A. Simmons Memorial Scholarship, 2021

\end{document}
